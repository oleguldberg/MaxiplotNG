\documentclass[11pt,a4paper]{article}
\usepackage[ansinew]{inputenc}
\usepackage{graphicx}
\usepackage{amsmath}
\usepackage[amsmath]{maxiplot}

\title{Maxiplot: Maxima and Gnuplot in \LaTeX.\\}
\date{\today}
\author{Ole Guldberg}

\def\Maxima{\emph{Maxima}}
\def\Gnuplot{\emph{Gnuplot}}

\begin{document}
\maketitle

\section{Introduction}
For those who do not know \Maxima, it is a symbolic calculation
program which can be used to compute derivatives and integrals, solve
equations, find limits, work with vectors and matrices and create
graphics, among many other things. It also adds the possibility to
write programs, thus expanding its capabilities. As if all this was
not enough, it is also released under the GNU General Public License
and it can be downloaded for free at
\texttt{http://maxima.sourceforge.net}, where there is also
documentation in several languages (including Spanish).

The purpose of this \LaTeX{} package is to provide ``programming''
capabilities importing the results, without the need of working with
various files and interfaces. Maxima code can be included within the
\LaTeX{} document. When the document is processed, a file with
extension \texttt{.mac} is generated, which can be directly processed
by Maxima, creating another file with extension \texttt{.mxp}; when
the \LaTeX{} document is processed again, that file will be
automatically inserted.

\Gnuplot{} commands can also be inserted, thanks to some additional
commands added by J. M. Mira. Thus, in addition to the auxiliary files
already mentioned, another file with extension \texttt{.gnp} will be
created, which after being processed by \Gnuplot{} can be added to the
document.



\section{Gnuplot}
While \Maxima{} can create graphics via \Gnuplot{}, sometimes it might
be preferable to work directly with this last program. In order to do
that, the environments \texttt{gnuplot} and its verbatim version
\texttt{vgnuplot} are used.

\begin{gnuplot}
set term png crop enhanced font "calibri, 10"
set output "t.png"
set yrange [-2:2]
set xrange [-4:4]
set xzeroaxis
set yzeroaxis
plot [-4:4] sin(x),cos(x)
\end{gnuplot}
\begin{center}
  \mxpIncludegraphics[scale=0.75]{t.png}
\end{center}

Here is a 3D example
\begin{verbatim}
\begin{gnuplot}
  set term png crop enhanced font "calibri, 10"
  set output "toros.png"
  set parametric
  set urange [0:2*pi]
  set vrange [-pi:pi]
  set isosamples 36,24
  set hidden3d
  set view 75,15,1,1
  unset key
  set ticslevel 0
  x1(u,v)=cos(u)+1*cos(u)*cos(v)
  y1(u,v)=sin(u)+1*sin(u)*cos(v)
  z1(u,v)=.5*sin(v)
  x2(u,v)=1+cos(u)+.5*cos(u)*cos(v)
  y2(u,v)=.5*sin(v)
  z2(u,v)=sin(u)+.5*sin(u)*cos(v)
  set multiplot
  splot x1(u,v), y1(u,v), z1(u,v) w pm3d, x2(u,v), y2(u,v), z2(u,v) w pm3d
  splot x1(u,v), y1(u,v), z1(u,v) lt 3,   x2(u,v), y2(u,v), z2(u,v) lt 5 
\end{gnuplot}
\begin{center}
  \mxpIncludegraphics[scale=0.75]{toros.png}
\end{center}
\end{verbatim}


\begin{gnuplot}
  set term png crop enhanced font "calibri, 10"
  set output "toros.png"
  set parametric
  set urange [0:2*pi]
  set vrange [-pi:pi]
  set isosamples 36,24
  set hidden3d
  set view 75,15,1,1
  unset key
  set ticslevel 0
  x1(u,v)=cos(u)+1*cos(u)*cos(v)
  y1(u,v)=sin(u)+1*sin(u)*cos(v)
  z1(u,v)=.5*sin(v)
  x2(u,v)=1+cos(u)+.5*cos(u)*cos(v)
  y2(u,v)=.5*sin(v)
  z2(u,v)=sin(u)+.5*sin(u)*cos(v)
  set multiplot
  splot x1(u,v), y1(u,v), z1(u,v) w pm3d, x2(u,v), y2(u,v), z2(u,v) w pm3d
  splot x1(u,v), y1(u,v), z1(u,v) lt 3,   x2(u,v), y2(u,v), z2(u,v) lt 5 
\end{gnuplot}
\begin{center}
  \mxpIncludegraphics[scale=0.75]{toros.png}
\end{center}

Let us examine the \verb|\mxpIncludegraphics| command: its usage is
the same as \verb|includegraphics| from package \verb|graphicx|; in
fact, it just makes sure that the graphic file exists before invoking
that macro.

\subsection{Problems}
This is an experimental version; many of the capabilities of Maxima
have not been tested and it has not been tried with the most important
\LaTeX{} packages. Thus, it will surely need some tweaking.

However, I think that most of the problems will appear when showing
certain outputs. For instance, if the result of a computation is too
long, it will not be easy to break it into several lines (except if
one works in Maxima and then copies the result to the document, of
course).

Other possible problems can be addressed from within the \LaTeX{}
document. By default, Maxima orders expressions by inverse
alphabetical order; hence, if we type:\\
\verb|  $$\imaxima{tex(x+y+z+t=0)}$$|\\
we get:
$$\imaxima{tex(x+y+z+t=0)}$$

That can be avoided by using Maxima functions \texttt{ordergreat} and
\texttt{unorder}:
\begin{verbatim}
\imaximacmd{ordergreat(x,y,z,t)$}
$$\imaxima{tex(x+y+z+t=0)}$$
\imaximacmd{unorder()$}
\end{verbatim}
%\imaximacmd{ordergreat(x,y,z,t)$}
%$$\imaxima{tex(x+y+z+t=0)}$$
%\imaximacmd{unorder()$}

Furthermore, if we would like to align several equations, we will need
to dive a little deeper:
\begin{verbatim}
\begin{maximacmd}
  ordergreat(x,y,z)$
  :lisp(defprop mequal (&=) texsym)
\end{maximacmd}

\begin{maxima*}
  eq1:a-2*b=x+y,
  eq2:b=2*x-3*y+2*z,
  tex(eq1),
  print("\\\\"),
  tex(eq2)
\end{maxima*}

\begin{maximacmd}
  unorder()$
  :lisp(defprop mequal (=) texsym)    
\end{maximacmd}
\end{verbatim}

\begin{maximacmd}
  ordergreat(x,y,z)$
  :lisp(defprop mequal (&=) texsym)
\end{maximacmd}

\begin{maxima*}
  eq1:a-2*b=x+y,
  eq2:b=2*x-3*y+2*z,
  tex(eq1),
  print("\\\\"),
  tex(eq2)
\end{maxima*}

\begin{maximacmd}
  unorder()$
  :lisp(defprop mequal (=) texsym)    
\end{maximacmd}

\begin{align}
  \maximacurrent
\end{align}

\section{A few last words}
As I mentioned before, this is an experimental package that will
probably need some amendments and additions, so any ideas or comments
will be welcome.
\\

\raggedleft{Jos\'e Miguel M. Planas}\\
\raggedleft{$<$nohaim@gmail.com$>$}\\[10pt]
\raggedleft{(English translation by Jaime Villate)}
                  
\end{document}