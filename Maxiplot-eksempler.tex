\documentclass[12pt,a4paper]{article}
\usepackage[utf8]{inputenc} 
\usepackage[danish]{babel} 
\usepackage[T1]{fontenc} 
\usepackage{graphicx}
\usepackage{amsmath}
\usepackage[amsmath]{maxiplot}

\def\Maxima{\emph{Maxima}}
\def\Gnuplot{\emph{Gnuplot}}

\begin{document}

\title{Eksempler på \Maxima{} og \Gnuplot{} i \LaTeX{}.}
\author{Ole Guldberg, ole@omgwtf.dk}
\date{\today}
\maketitle

I dette dokument vil der være eksempler på hvordan man kan benytte \Maxima{} og \Gnuplot{} i sine \LaTeX{}-dokumenter. 
\Maxima{} er et CAS-værktøj - Computer Algebraic System, det understøtter både numerisk og symbolske udregninger og \Gnuplot{} er et værktøj til plotning af grafer og funktioner. 
\\ \\
Man vil have mest gavn af disse eksempler, hvis man også ser på \LaTeX{}-kildekoden og de tilhørende .mac og .mxp-filer.

\section*{Simple udregninger}

$$\sqrt{13} = 
\begin{maxima}
  tex(float(sqrt(13)) )
\end{maxima}$$

$$\frac{\sqrt{13}}{3}=
\begin{maxima}
  tex(float(sqrt(13)/3))
\end{maxima}$$

$$\cos(3) = 
\begin{maxima}
  tex(float(cos(3)) )
\end{maxima}$$

\section*{Et eksempel på løsning af en andengradsligning}

Andengradsligningen: $$-2 \cdot x^2+2 \cdot x+5 = 0$$ har løsningerne:
$$\begin{maxima}
  tex(solve(-2*x^2+2*x+5=0,x))
\end{maxima}
$$

\section*{Et eksempel på løsning skæringspunkt mellem linie og parabel}

Skæring mellem parabel og linie kan ses som en ligning hvor man sætte parablen og liniens ligning lig hinanden.
$$x^2 - 2 \cdot x + 3 = - 3 \cdot x +7$$
Parablen og linien skærer i
$$\begin{maxima}
  tex(solve(x^2-2*x+3=-3*x+7,x))
\end{maxima}$$

\section*{Integraler}

$$\int x^2 + 2 \cdot x  -11 dx = 
\begin{maxima}
 tex(integrate(x^2+2*x-11,x))
\end{maxima} + K$$
 
$$\begin{maxima}
  f: x/(x^3-3*x+2),
  tex('integrate(f,x)),
  print("="),
  tex(integrate(f,x)),
  print("+K")
\end{maxima}$$

\section*{Praktiske informationer og fif}

\begin{itemize}
  \item{Hvis du vil se svaret på \Maxima{}s udregninger i \LaTeX{}-format skal du kikke i filen: Maxiplot-eksempler.mxp.}
  \item{Undgå filnavne med danske karaktere og mellemrum.}
\end{itemize}  

\end{document}